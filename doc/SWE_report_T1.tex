\documentclass{article}
\usepackage[italian]{babel}
\usepackage{hyperref}

\title{VezGammon - Report finale Team 1 \\ \large Progetto di Ingegneria del Software}
\author{Scrum Master - Lorenzo Peronese}

\begin{document}

\maketitle

\tableofcontents

\newpage

\section{Descrizione del prodotto}
\subsection{Scope}
Per il corso di Ingegneria del Software, il team ha scelto di lavorare a VezGammon, un'applicazione web di backgammon. Il progetto ha diversi riferimenti 
alla città di Bologna, in cui è nato e si è sviluppato: in primis ovviamente il nome ("vez" viene usato a Bologna come intercalare e significa "amico/fratello"); 
inoltre una volta nel sito il cursore diventa un simpatico tortellino e la board utilizza i colori rosso e blu, rappresentativi della città. \\
L'applicazione offre un'esperienza di gioco completa con diverse modalità. Gli utenti possono sfidarsi in partite locali sullo stesso dispositivo, mettere alla prova 
le proprie abilità contro intelligenze artificiali di vari livelli di difficoltà, o confrontarsi online con altri giocatori attraverso un sistema di matchmaking 
o mediante link di invito diretti. \\
È inoltre possibile organizzare tornei a quattro partecipanti, combinando liberamente giocatori reali e agenti artificiali. \\
Il sistema di classificazione si basa sul metodo Elo, ampiamente utilizzato negli scacchi. Ogni nuovo account parte da una base di 800 punti, che vengono poi 
aggiornati dopo ogni partita online o torneo. Il calcolo dell'aggiornamento tiene conto di diversi fattori: l'esito della partita, l'utilizzo del dado double 
e il punteggio Elo di entrambi i giocatori. Questo sistema garantisce una competizione equilibrata e permette di mantenere una classifica globale, 
dove ogni giocatore può aspirare a raggiungere le posizioni più alte. \\
VezGammon include anche ricche funzionalità social e di progressione. Gli utenti possono consultare (e condividere sui social) in ogni momento statistiche dettagliate delle proprie partite, 
visualizzare grafici che mostrano l'andamento del proprio Elo nel tempo e analizzare i match recenti. Il sistema premia inoltre i giocatori con speciali badge 
al raggiungimento di specifici obiettivi, come vincere un determinato numero di partite o raggiungere certi livelli di punteggio Elo, aggiungendo un ulteriore 
elemento di coinvolgimento e progressione al gioco. \\

\subsection{Backlog}

\subsection{Casi d'uso}

\subsection{Architettura}

\section{Sprint 0}

\section{Sprint 1}

\subsection{Goal}

\subsection{Backlog}

\subsection{Incremento}

\subsection{Definition of Done}

\subsection{Esempio di test fatti}

\subsection{Burndown}

\subsection{Retrospettiva}

\section{Sprint 2}

\subsection{Goal}

\subsection{Backlog}

\subsection{Incremento}

\subsection{Definition of Done}

\subsection{Esempio di test fatti}

\subsection{Burndown}

\subsection{Retrospettiva}

\section{Sprint 3}

\subsection{Goal}

\subsection{Backlog}

\subsection{Incremento}

\subsection{Definition of Done}

\subsection{Esempio di test fatti}

\subsection{Burndown}

\subsection{Retrospettiva}

\section{Sprint 4}

\subsection{Goal}

\subsection{Backlog}

\subsection{Incremento}

\subsection{Definition of Done}

\subsection{Esempio di test fatti}

\subsection{Burndown}

\subsection{Retrospettiva}

\section{Release sprint}

\section{Descrizione del processo}

4 sprint da 2 settimane ciascuno

\subsection{Il team}

\subsection{Teambuilding}

\subsection{Gitinspector}

\subsection{Strumenti di comunicazione}

\subsection{Uso di LLM}

\subsection{Deployment del prodotto}

\subsection{Retrospettiva finale con Essence}

\section{Demo 3 min}
Link qui

\section{Artefatti realizzati}

\subsection{Di processo}

\subsubsection{GitLab}

\subsubsection{Taiga}

\subsubsection{Mattermost}

\subsubsection{Jenkins}

\subsubsection{SonarQube}

\subsubsection{Status}

\subsection{Di prodotto}
Server di produzione e server develop per il testing


\end{document}
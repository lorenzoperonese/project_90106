\documentclass{article}
\usepackage{graphicx} % Necessario per inserire immagini

\title{Vezgammon - Definizione di Completamento}
\author{Product Owner: Diego Barbieri}
\date{Ottobre 2024}

\begin{document}

\maketitle

\section{Requisiti della Code Base}

\subsection{Qualità del Codice}
\begin{itemize}
    \item Tutto il codice deve seguire le convenzioni di denominazione in camelCase.
    \item Ogni modulo deve avere e superare i propri test unitari (e mock, se sviluppato con TDD).
    \item Nessun codice critico o errori gravi rilevati dagli strumenti di analisi statica (es. SonarQube).
    \item Tutte le API devono fornire un'anteprima con Swagger e includere test specifici per le API.
    \item Per le principali pagine del front-end, la funzionalità di base deve essere testata tramite test front-end.
\end{itemize}

\subsection{Documentazione}
\begin{itemize}
    \item Documentazione per ogni modulo: un'anteprima markdown dell'interfaccia allineata con il diagramma UML delle classi.
    \item README principale aggiornato con le funzionalità più recenti.
\end{itemize}

\subsection{Merge Request}
\begin{itemize}
    \item Ogni merge request (MR) deve essere approvata dal Scrum Master (SM) o dal Product Owner (PO).
    \item La MR deve superare tutti i test nella pipeline di testing di Jenkins.
\end{itemize}

\subsection{Interfaccia Utente}
\begin{itemize}
    \item Soddisfa gli standard WCAG 2.1 AA: compatibile con tutti i browser moderni.
    \item Responsiva su laptop, tablet e smartphone (solo visualizzazione verticale).
\end{itemize}

\subsection{Prestazioni}
\begin{itemize}
    \item L'app mantiene prestazioni accettabili (tempo di caricamento inferiore a 2 secondi per l'interfaccia principale del gioco).
\end{itemize}

\section{Requisiti Funzionali}

\subsection{Interfaccia Utente}
\begin{itemize}
    \item La scacchiera viene visualizzata correttamente su tutte le dimensioni di schermo standard (desktop, tablet, mobile).
    \item I pezzi, i dadi e gli altri elementi di gioco sono visivamente chiari e rispondono alle interazioni dell'utente.
    \item Supporto per modalità scura e chiara, se specificato.
    \item Le animazioni e le transizioni sono responsive e rispettano gli standard di UX.
\end{itemize}

\subsection{Logica di Gioco}
\begin{itemize}
    \item Gli utenti possono avviare una nuova partita e scegliere il tipo di avversario (IA o altro giocatore).
    \item I giocatori possono lanciare i dadi e muovere i pezzi secondo le regole del Backgammon.
    \item Il gioco applica correttamente le regole del Backgammon, inclusi:
    \begin{itemize}
        \item \textbf{Impostazione:} Configurazione iniziale e movimento.
        \item \textbf{Azioni:} Cattura e rilascio dei pezzi.
        \item \textbf{Vittoria:} Condizioni di vittoria (es. rimozione di tutti i pezzi).
    \end{itemize}
    \item È disponibile uno storico o log di gioco che mostra le mosse effettuate da ciascun giocatore.
\end{itemize}

\section{Requisiti di Deployment}

\subsection{Compatibilità Ambientale}
\begin{itemize}
    \item L'applicazione funziona correttamente negli ambienti di sviluppo e produzione.
    \item L'applicazione deve richiedere un file di configurazione, facile da configurare e compatibile con l'architettura di deployment.
\end{itemize}

\subsection{Build e Rilascio}
\begin{itemize}
    \item L'applicazione può essere compilata senza errori o avvisi.
    \item L'app è configurata per l'integrazione e il deployment continui, con test automatizzati per ogni merge.
\end{itemize}

\subsection{Monitoraggio e Logging}
\begin{itemize}
    \item Sono attivi il monitoraggio degli errori in tempo reale e il logging (es. con Sentry).
    \item I log catturano eventi critici come l'inizio delle partite, le disconnessioni e gli errori dell'utente.
\end{itemize}

\end{document}

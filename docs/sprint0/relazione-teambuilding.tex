\documentclass{article}
\usepackage{amsmath}
\usepackage[utf8]{inputenc}

\title{\textbf{Attività di Team Building:\\Scrumble Game \& Escape the Boom!}}
\author{Team 1\\
Lorenzo Peronese, Diego Barbieri, Samuele Musiani,\\
Fabio Murer, Emanuele Argonni, Omar Ayache}
\date{23/10/2024}

\begin{document}

\maketitle

\section*{Parte 1: Scrumble Game}

\section*{Introduzione}
Il team ha partecipato a una sessione di \textit{Scrumble Game}, un'attività formativa basata sulla metodologia Scrum. Lo \textbf{Scrum Master} ha iniziato illustrando le regole del gioco, seguito dal \textit{Project Owner} che ha presentato il prodotto da sviluppare.

\section*{Composizione del Team}
\begin{itemize}
    \item \textbf{Scrum Master}
    \begin{itemize}
        \item Lorenzo Peronese
    \end{itemize}
    \item \textbf{Project Owner}
    \begin{itemize}
        \item Diego Barbieri
    \end{itemize}
    \item \textbf{Development Team}
    \begin{itemize}
        \item Samuele Musiani
        \item Fabio Murer
        \item Emanuele Argonni
        \item Omar Ayache
    \end{itemize}
\end{itemize}

\section*{Il Progetto}
Il team ha sviluppato un progetto basato sullo scenario E-Commerce, con una rielaborazione creativa del Project Owner che lo ha trasformato in un \textbf{"marketplace di auto rubate"} che permettesse la vendita attraverso il \textbf{dark web}. Questa personalizzazione ha permesso di mantenere gli elementi base dell'E-Commerce aggiungendo sfide fuori dal comune, rendendo l'attività più divertente e coinvolgente. Alcune user story originali sono state adattate al nuovo contesto, per renderle coerenti con le nuove specifiche:

\begin{itemize}
    \item Da \textit{"Come cliente, voglio poter creare un account utente"} a \textbf{"Come cliente, voglio poter comprare anonimamente"} (2 punti, L),
    \item Da \textit{"Come cliente, voglio poter pagare con Paypal, Visa o bonifico"} a \textbf{"Come cliente, voglio poter pagare con criptovalute non tracciate"} (4 punti, L),
    \item Da \textit{"Come venditore, voglio evitare nuovi ordini durante le mie vacanze"} a \textbf{"Come venditore, voglio evitare nuovi ordini durante controlli delle forze dell'ordine"} (1 punto, S).
\end{itemize}

\section*{Cronologia degli Sprint}

Il gioco si è articolato in \textbf{sprint}, ognuno dei quali aveva associato un numero di \textbf{user stories} (US). L'obiettivo era completare il maggior numero possibile di task e mantenere sotto controllo il debito tecnico.

\subsection*{Sprint 1}
Il team ha affrontato le user stories US n.5 da 16 task, US n.11 da 48 task e US 13 da 80 task. Durante il primo giorno, i membri del team hanno effettuato lanci di dadi che hanno definito il loro avanzamento iniziale, con risultati altalenanti, dovuti anche alla poca considerazione portata agli imprevisti. Nel secondo giorno, si è verificata una situazione di debito tecnico significativo, con alcuni lanci di dadi che hanno peggiorato la situazione, ma il team è riuscito a completare la US n.5. Nei giorni successivi, problemi imprevisti hanno continuato ad accumulare debiti e rallentare il progresso, mentre qualche task è stato completato grazie a decisioni strategiche e lanci favorevoli. Alla fine dello sprint, il team ha completato solo due delle tre user stories previste, con 80 task totali completati su 140. 

\subsubsection*{Sprint Review}  
La review ha evidenziato che l'assegnazione dei task era mal gestita, con una sovrastima del lavoro e un mancato adeguamento ai problemi emersi. Ciò ha portato a una bassa efficienza complessiva e a un debito tecnico piuttosto elevato.

\subsubsection*{Retrospettiva}  
Si è discusso dei conflitti interni tra Omar e Emanuele, mentre Fabio ha cercato di riportare la calma. È stata proposta l'introduzione di un dado "truccato" per migliorare il morale. Samuele ha espresso indifferenza, ma ha riconosciuto la necessità di migliorare l’organizzazione del team.


\subsection*{Sprint 2}
\textbf{User Story:} US 1 (96 task)\\
\textbf{Debito arretrato:} 28 punti\\
\textbf{Risultati:} User story completata integralmente\\
\textbf{Miglioramenti:} Riduzione significativa del debito tecnico, migliore gestione delle priorità e risoluzione dei conflitti interni al gruppo.

\subsection*{Sprint 3}
\textbf{User Story:} US 7 (130 task incluse penalità)\\
\textbf{Debito arretrato:} 9 punti\\
\textbf{Risultati:} I task sono stati completati con successo, la coesione del team viene di nuovo messa in discussione a causa di un presunto comportamento scorretto da parte di un developer, ma la situazione viene subito chiarita.

\subsection*{Sprint 4}
\textbf{User Stories:} US 9 (75 task), us 12 (110 task)\\
\textbf{Debito arretrato:} 12 punti\\
\textbf{Risultati:} Completata 1 user stories (120/185 task) sulle due complessive, il debito tecnico continua a ostacolare il progresso; si discute riguardo a una migliore suddivisione dei ruoli e alla diminuzione del carico di lavoro delle user stories per evitare l'accumulo di debito.

\subsection*{Sprint 5}
\textbf{User Stories:} US 14 (100 task), US 15 (90 task)\\
\textbf{Debito arretrato:} 6 punti\\
\textbf{Risultati:} Sprint concluso con successo con entrambe le user stories completate, maggior efficacia nella gestione di task e debito arretrato, il miglioramento organizzativo ha permesso di superare le difficoltà iniziali concludendo il ciclo di sviluppo con successo, anche se il progetto iniziale è stato ridimensionato: il team non è riuscito a sviluppare diverse feature relative alle user stories non affrontate.

\section*{Obiettivo Finale}
Lo \textit{Scrumble Game} ha permesso al team di comprendere le difficoltà di gestione di un'attività complessa, affrontando il tutto con serietà e un pizzico di divertimento. 
Dopo una fase iniziale di difficoltà necessaria per comprendere appieno il gioco e le tattiche migliori, il team ha capito il funzionamento ed è riuscito a proseguire al meglio l'attività.
Il gioco ha pienamente centrato l'obiettivo di far comprendere ad ogni singolo membro della squadra che nel modello agile è fondamentale che ogni individuo contribuisca attivamente al raggiungimento degli obiettivi comuni, attraverso la collaborazione, la trasparenza e l'adattabilità.


\section*{Parte 2: Escape the Boom!}

\section*{Introduzione}
Come seconda attività di team building, il gruppo ha affrontato \textbf{Escape the Boom!}, un gioco cooperativo che simula il disinnesco di un ordigno, richiedendo comunicazione efficace e rapidità decisionale.

\section*{Struttura dell'Attività}
\begin{itemize}
    \item \textbf{Durata:} 20 minuti
    \item \textbf{Composizione:} Due squadre da 3 persone
    \item \textbf{Ruoli:}
    \begin{itemize}
        \item \textbf{Team Manuale:} Interpretazione e comunicazione delle istruzioni (Diego Barbieri, Fabio Murer, Omar Ayache)
        \item \textbf{Team Operativo:} Esecuzione delle procedure di disinnesco (Lorenzo Peronese, Samuele Musiani, Emanuele Argonni)
    \end{itemize}
    \item \textbf{Obiettivo:} Completare il maggior numero di livelli entro il tempo limite
\end{itemize}

\section*{Risultati}
Il team ha completato con successo 5 livelli nei 20 minuti disponibili, dimostrando una progressiva ottimizzazione della comunicazione e dei processi decisionali.
Come per l'attività precedente, anche qui inizialmente il gruppo ha avuto difficoltà a completare i livelli velocemente, ma una volta riconosciuti i pattern ricorrenti i tempi si sono accorciati e la comunicazione è diventata sempre più breve e efficace, consentendo il completamento dei livelli in poco tempo.

\section*{Analisi Comparativa delle Due Attività}

\subsection*{Punti di Forza Emersi}
\begin{itemize}
    \item \textbf{Comunicazione:}
    \begin{itemize}
        \item Scrumble Game: Sviluppo di comunicazione strutturata e pianificazione
        \item Escape the Boom!: Affinamento della comunicazione rapida e precisa
    \end{itemize}
    
    \item \textbf{Gestione del Tempo:}
    \begin{itemize}
        \item Scrumble Game: Pianificazione a lungo termine e gestione degli sprint
        \item Escape the Boom!: Ottimizzazione delle decisioni sotto pressione
    \end{itemize}
    
    \item \textbf{Collaborazione:}
    \begin{itemize}
        \item Scrumble Game: Definizione chiara dei ruoli e responsabilità
        \item Escape the Boom!: Fiducia reciproca e adattamento rapido ai cambiamenti
    \end{itemize}
\end{itemize}

\section*{Conclusioni}
Le due attività hanno contribuito in modo complementare alla crescita del team:

\begin{itemize}
    \item \textbf{Scrumble Game} ha sviluppato:
    \begin{itemize}
        \item Capacità di pianificazione strategica
        \item Gestione del debito tecnico
        \item Risoluzione strutturata dei conflitti
    \end{itemize}
    
    \item \textbf{Escape the Boom!} ha potenziato:
    \begin{itemize}
        \item Comunicazione rapida ed efficace
        \item Gestione dello stress
        \item Adattabilità ai cambiamenti
    \end{itemize}
\end{itemize}

La combinazione delle due esperienze ha creato un percorso completo, bilanciando aspetti di pianificazione a lungo termine con capacità di risposta rapida alle emergenze, contribuendo significativamente alla coesione del team.

\end{document}

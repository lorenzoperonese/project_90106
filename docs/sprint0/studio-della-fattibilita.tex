\documentclass{article}

\title{Studio di Fattibilità per il Progetto di Ingegneria del Software}

\date{October 2024}

\begin{document}

\maketitle

\section{Obiettivi del progetto:}

\begin{enumerate}
    \item    Creare una web app in cui gli utenti possano giocare a backgammon online.
    \item    Seguire la metodologia Agile, con consegne ogni 2 settimane.
    \item    Utilizzare Go per il backend, Vue per il frontend, un database relazionale e Docker per la containerizzazione.
\end{enumerate}

\section{Vincoli temporali:}

\begin{enumerate}
    \item   Durata: 1 mese e mezzo circa (6 settimane).
    \item   Team: 6 persone.
    \item   Sprint: Ogni 2 settimane (3 sprint in totale).
    \item   Consegne: Ogni sprint deve concludersi con le funzionalità assegnate nelle user stories o un set di funzionalità completate.
\end{enumerate}

\section{Analisi tecnica:}

\begin{enumerate}
    \item Backend in Go: Go è adatto per applicazioni backend grazie alle sue prestazioni e alla gestione della concorrenza, ideale per gestire le dinamiche di gioco in tempo reale come il backgammon.
    \item Frontend in Vue: Vue è un framework reattivo perfetto per creare interfacce utente dinamiche, ideale per l'interfaccia del gioco.
    \item Database relazionale: sarà utilizzato per memorizzare utenti, partite, risultati e progressi dei giocatori.
\end{enumerate}

\section{Risorse necessarie:}

\begin{enumerate}
    \item Competenze del team:
        \begin{enumerate}
            \item 2 sviluppatori backend (Go).
            \item 2 sviluppatori frontend (Vue).
            \item 1 esperto di database.
            \item 1 tester/coordinatore Agile.
            \item 1 Scrum Master per il coordinamento dei developer.
            \item 1 Product Owner per gestire il rapporto con gli Stakeholders
        \end{enumerate}
    \item Strumenti:
        \begin{enumerate}
            \item GitLab per la gestione del repository.
            \item Docker per la containerizzazione.
            \item Jenkins per l'Integrazione Continua (CI) e il deployment.
            \item SonarQube per garantire la qualità del codice e rispettare le convenzioni.
            \item Ambienti di sviluppo per Go e Vue.
        \end{enumerate}
\end{enumerate}

\section{Fattori di rischio:}

\begin{enumerate}
    \item \textbf{Coordinamento:} Il team dovrà assicurarsi di rispettare le scadenze e sincronizzare il lavoro tra tutte le parti. Ritardi in un'area (es. backend) potrebbero influire sulle altre (es. frontend).
   \item \textbf{Competenze tecniche:} Assicurarsi che tutti i developers abbiano familiarità con le tecnologie scelte (Go, Vue) in base al loro ruolo interno. Se qualcuno non ha esperienza, potrebbe essere necessario del tempo extra.
   \item \textbf{Integrazione:} L'integrazione tra frontend e backend potrebbe richiedere più tempo del previsto se emergono bug o incompatibilità.
   \item \textbf{Utilizzo di nuovi strumenti:} L'utilizzo di Jenkins, Taiga e SonarQube ha sicuramente benefici a lungo andare, ma nel breve termine è necessario spendere tempo per configurarli e comprendere il loro funzionamento.
   \item \textbf{Gestione del tempo:} Riuscire a rispettare le consegne ogni 2 settimane mantenendo alta la qualità del codice e delle prestazioni è una sfida. Se delle task di uno sprint non vengono completate, questo potrebbe avere effetti a cascata sugli sprint successivi.
   \item \textbf{Esperienza utente:} Durante lo sviluppo dell'interfaccia in Vue, garantire interazioni fluide e la gestione dinamica del gioco in tempo reale (es. azioni a turni, aggiornamenti dello stato del gioco) sarà un fattore cruciale per il successo del progetto.
\end{enumerate}

\section{Conclusione:}
    Per riuscire a concludere con successo questo progetto sarà fondamentale la coordinazione tra i membri del team, cercando di evitare il più possibile conflitti interni.
    
    Ogni membro del gruppo dovrà impegnarsi per completare le sue attività, ma allo stesso tempo essere disponibile ad aiutare i colleghi in caso di necessità.
    
    L'uso di molti software nuovi è una sfida importante: sarà cruciale affrontarla nel modo corretto e dedicare del tempo iniziale alla formazione e alla configurazione degli strumenti per evitare rallentamenti nelle fasi successive del progetto.
    
    Con una gestione attenta dei rischi identificati e un team coeso, il progetto ha buone probabilità di essere completato con successo nei tempi previsti.
\end{document}

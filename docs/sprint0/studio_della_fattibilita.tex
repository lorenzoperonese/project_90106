\documentclass{article}
\usepackage{graphicx} % Required for inserting images

\title{Studio di Fattibilità per il Progetto di Ingegneria del Software}

\date{October 2024}

\begin{document}

\maketitle

\section{Obiettivi del progetto:}

\begin{enumerate}
    \item    Creare una web app in cui gli utenti possano giocare a backgammon online.
    \item    Seguire la metodologia Agile, con consegne ogni 2 settimane.
    \item    Utilizzare Go per il backend, Vue per il frontend e SQL per il database.
\end{enumerate}

\section{Vincoli temporali:}

\begin{enumerate}
    \item   Durata: 1 mese e mezzo ca (6 settimane).
    \item   Team: 6 persone.
    \item   Sprint: Ogni 2 settimane (3 sprint in totale).
    \item   Consegne: Ogni sprint deve concludersi con le funzionalità assegnate nelle user stories o un set di funzionalità completate.
\end{enumerate}

\section{Analisi tecnica:}

\begin{enumerate}
    \item Backend in Go: Go è ben adatto per applicazioni backend grazie alle sue prestazioni e alla gestione della concorrenza, ideale per gestire le dinamiche di gioco in tempo reale come il backgammon.
    \item Frontend in Vue: Vue è un framework reattivo perfetto per creare interfacce utente dinamiche, ideale per l'interfaccia del gioco.
    \item Database SQL: SQL sarà utilizzato per memorizzare utenti, partite, risultati e progressi dei giocatori.
\end{enumerate}

\section{Risorse necessarie:}

\begin{enumerate}
    \item Competenze del team:
        \begin{enumerate}
            \item 2 sviluppatori backend (Go).
            \item 2 sviluppatori frontend (Vue).
            \item 1 esperto di database (SQL).
            \item 1 tester/coordinatore Agile.
            \item 1 Scrum Master per il coordinamento dei developer.
            \item 1 Product Owner per gestire il rapporto con gli Stakeholders
        \end{enumerate}
    \item Strumenti:
        \begin{enumerate}
            \item GitLab per la gestione del repository.
            \item Docker per il deployment.
            \item Jenkins per l'Integrazione Continua (CI).
            \item SonarQube per garantire la qualità del codice e rispettare le convenzioni.
            \item Ambienti di sviluppo per Go, Vue e SQL.
        \end{enumerate}
\end{enumerate}

\section{Fattori di rischio:}

\begin{enumerate}
    \item Coordinamento: Il team dovrà assicurarsi di rispettare le scadenze e sincronizzare il lavoro tra tutte le parti. Ritardi in un'area (es. backend) potrebbero influire sulle altre (es. frontend).
   \item Competenze tecniche: Assicurarsi che tutti i membri abbiano familiarità con le tecnologie scelte (Go, Vue, SQL). Se qualcuno non ha esperienza, potrebbe essere necessario del tempo extra per apprendere.
   \item Integrazione: L'integrazione tra frontend e backend potrebbe richiedere più tempo del previsto se emergono bug o incompatibilità.
   \item Deployment e CI: L'implementazione di Docker per il deployment e Jenkins per la CI aggiunge complessità. Garantire che la pipeline di integrazione continua funzioni senza intoppi potrebbe richiedere sforzi extra.
   \item Qualità e standard: Seguire gli standard di codifica con SonarQube richiederà attenzione ai dettagli e potrebbe richiedere modifiche se il codice non rispetta subito gli standard.
   \item Gestione del tempo: Riuscire a rispettare le consegne ogni 2 settimane mantenendo alta la qualità del codice e delle prestazioni è una sfida. Se un obiettivo di sprint non viene raggiunto, potrebbe avere effetti a cascata sugli sprint successivi.
   \item Esperienza utente: Durante lo sviluppo dell'interfaccia in Vue, garantire interazioni fluide e la gestione dinamica del gioco in tempo reale (es. azioni a turni, aggiornamenti dello stato del gioco) sarà un fattore cruciale per il successo del progetto.
\end{enumerate}

\section{Conclusione:}
    Il progetto è fattibile entro i 3 Sprint previsti grazie alla suddivisione del lavoro e al numero di persone nel team. Tuttavia, sarà fondamentale coordinare bene il lavoro tra frontend, backend e database per rispettare le scadenze. 
   L'uso di GitLab, Taiga, Docker, Jenkins e SonarQube aggiunge complessità, ma garantisce una solida base per il deployment, l'integrazione continua e la verifica della qualità del codice.

\end{document}
